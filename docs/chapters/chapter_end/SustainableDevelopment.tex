% Anexo: Relación del trabajo con los Objetivos de Desarrollo Sostenible de la Agenda 2030
\chapter*{Anexo I. Relación del trabajo con los Objetivos de Desarrollo Sostenible de la Agenda 2030}
\addcontentsline{toc}{chapter}{Anexo I. Relación del trabajo con los Objetivos de Desarrollo Sostenible de la Agenda 2030}

En este anexo se presenta una tabla que relaciona el proyecto "Neural Analytics" con los Objetivos de Desarrollo Sostenible (ODS) de la Agenda 2030. Esta relación se ha evaluado en términos de su impacto potencial, clasificando cada ODS en categorías de alto, medio, bajo o no procede según su relevancia para el proyecto.

\begin{longtable}{|p{8cm}|p{1.5cm}|p{1.5cm}|p{1.5cm}|p{1.5cm}|}
\hline
\textbf{Objetivos de Desarrollo Sostenibles} & \textbf{Alto} & \textbf{Medio} & \textbf{Bajo} & \textbf{No Procede} \\
\hline
\endfirsthead

\hline
\textbf{Objetivos de Desarrollo Sostenibles} & \textbf{Alto} & \textbf{Medio} & \textbf{Bajo} & \textbf{No Procede} \\
\hline
\endhead

\hline
\endfoot

\hline
\endlastfoot

ODS 1. Fin de la pobreza. & & & & X \\
\hline

ODS 2. Hambre cero. & & & & X \\
\hline

ODS 3. Salud y bienestar. & X & & & \\
\hline

ODS 4. Educación de calidad. & & & & X \\
\hline

ODS 5. Igualdad de género. & & & & X \\
\hline

ODS 6. Agua limpia y saneamiento. & & & & X \\
\hline

ODS 7. Energía asequible y no contaminante. & & & & X \\
\hline

ODS 8. Trabajo decente y crecimiento económico. & & & & X \\
\hline

ODS 9. Industria, innovación e infraestructuras. & X & & & \\
\hline

ODS 10. Reducción de las desigualdades. & X & & & \\
\hline

ODS 11. Ciudades y comunidades sostenibles. & & & & X \\
\hline

ODS 12. Producción y consumo responsables. & & & & X \\
\hline

ODS 13. Acción por el clima. & & & & X \\
\hline

ODS 14. Vida submarina. & & & & X \\
\hline

ODS 15. Vida de ecosistemas terrestres. & & & & X \\
\hline

ODS 16. Paz, justicia e instituciones sólidas. & & & & X \\
\hline

ODS 17. Alianzas para lograr objetivos. & & & & X \\
\hline

\end{longtable}

\newpage

\section*{Relación con los Objetivos de Desarrollo Sostenible}

En el contexto del proyecto "Neural Analytics", se ha identificado una relación significativa con varios objetivos de la Agenda 2030 para el Desarrollo Sostenible. La naturaleza innovadora y tecnológica del proyecto, junto con sus potenciales aplicaciones en el ámbito de la salud y la accesibilidad, permite establecer vínculos directos con los siguientes ODS:

\subsection*{ODS 3 (Salud y bienestar) - Relación ALTA}

El proyecto Neural Analytics presenta una aplicación directa en el campo de la tecnología sanitaria mediante el desarrollo de interfaces cerebro-computadora basadas en electroencefalografía (EEG). Esta tecnología tiene el potencial de mejorar significativamente la autonomía y calidad de vida de personas con diversidad funcional, especialmente aquellas con limitaciones de movilidad severas. La capacidad de controlar dispositivos externos a través de señales cerebrales representa un avance sustancial en tecnologías asistivas, contribuyendo directamente al objetivo de garantizar una vida sana y promover el bienestar para todos en todas las edades.

\subsection*{ODS 9 (Industria, innovación e infraestructuras) - Relación ALTA}

El desarrollo de Neural Analytics constituye un claro ejemplo de innovación tecnológica en el emergente campo de las interfaces cerebro-computadora. El proyecto incorpora tecnologías de vanguardia como redes neuronales profundas, procesamiento de señales en tiempo real y arquitecturas de software modulares y escalables. Esta contribución al avance tecnológico se alinea perfectamente con el objetivo de construir infraestructuras resilientes, promover la industrialización inclusiva y sostenible, y fomentar la innovación.

\subsection*{ODS 10 (Reducción de las desigualdades) - Relación ALTA}

Aunque la implementación actual del proyecto se encuentra en fase de prototipo, la tecnología desarrollada tiene el potencial a largo plazo de contribuir significativamente a la reducción de las brechas de accesibilidad. Las interfaces cerebro-computadora pueden democratizar el acceso a la tecnología para personas con limitaciones de movilidad severas, proporcionando nuevas formas de interacción e independencia. Sin embargo, se reconoce que esta contribución requiere de desarrollos posteriores y políticas de accesibilidad adecuadas para materializarse completamente.

