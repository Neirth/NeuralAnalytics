\chapter{Conclusiones}

Este trabajo ha sido largo. Empecé con una idea que tenía desde hace años y, al final, la he podido probar en la práctica. Marca el final de la carrera.

Ver los resultados me ha dejado bastante satisfecho. Lo que aprendí en clase y en prácticas lo he usado aquí. No es poca cosa. Me quedo con lo que he sacado y con ganas de seguir mirando temas de neurociencia. Es un campo que puede ayudar a mucha gente.

En lo técnico, el proyecto ha sido variado. He tenido que mirar neurociencia, entender cómo funciona la memoria del color, y también usar modelos de deep learning para señales EEG. El sistema está hecho con arquitectura hexagonal, que ayuda a mantener y escalar el código. Además, cumple con la UNE-EN 62304, que es importante en sanidad.

Los resultados del modelo y del sistema muestran que se puede usar una interfaz cerebro-computadora para controlar cosas en casa. Esto puede servir para que personas con problemas de movilidad tengan más autonomía. El sistema es modular y se puede adaptar a otros dispositivos.

A nivel personal, he aprendido a resolver problemas reales. No todo ha sido fácil, pero he tirado para adelante. Más allá de lo técnico, me quedo con haber probado un campo nuevo como la neurotecnología.

Sé que este trabajo no es una aportación grande al campo, pero me ha dado experiencia. Quizá en el futuro pueda hacer algo más serio en neurotecnología.

Con esto cierro una etapa. Ahora toca seguir aprendiendo y, si se puede, aportar algo útil a la tecnología para las personas.
