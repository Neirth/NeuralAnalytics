\chapter{Conclusiones}

Este proyecto ha representado un desarrollo técnico extenso que materializó conceptos teóricos en una implementación práctica funcional. Se logró validar la viabilidad de interfaces cerebro-computadora para aplicaciones domóticas, completando un ciclo completo de desarrollo desde la investigación inicial hasta la implementación en hardware especializado.

Los resultados obtenidos demuestran satisfacción respecto a los objetivos establecidos. La integración de conocimientos adquiridos durante la formación académica se aplicó efectivamente en un contexto práctico real. El proyecto establece una base sólida para futuras investigaciones en el campo de la neurotecnología aplicada.

Desde el punto de vista técnico, el proyecto integró múltiples disciplinas: neurociencia, procesamiento de señales cerebrales, y modelos de aprendizaje profundo para señales EEG. El sistema se construyó siguiendo arquitectura hexagonal, facilitando mantenimiento y escalabilidad del código. Además, se garantizó el cumplimiento de la normativa UNE-EN 62304, aspecto fundamental en el desarrollo de tecnología sanitaria.

Los resultados del modelo y del sistema demuestran la viabilidad de utilizar interfaces cerebro-computadora para control domótico. Esta tecnología presenta potencial significativo para mejorar la autonomía de personas con limitaciones de movilidad. El diseño modular del sistema permite adaptación a diferentes dispositivos y casos de uso.

A nivel de desarrollo profesional, el proyecto proporcionó experiencia valiosa en resolución de problemas técnicos complejos. La exposición a un campo emergente como la neurotecnología amplió considerablemente las perspectivas de aplicación tecnológica.

\newpage
\section{Trabajos Futuros}

Durante el desarrollo se identificaron múltiples oportunidades de mejora y extensión que constituyen líneas de investigación prometedoras:

\begin{itemize}
    \item \textbf{Ampliación del dataset}: Incorporar datos de múltiples usuarios permitiría mejorar la generalización del modelo y su aplicabilidad en diversos perfiles neurológicos, aumentando significativamente la robustez del sistema.
    
    \item \textbf{Expansión del espectro de detección}: Evolucionar del sistema actual de dos colores (rojo y verde) hacia un sistema de mayor variedad cromática, incrementando las posibilidades de control y precisión del sistema.
    
    \item \textbf{Integración con estándares emergentes}: Implementar compatibilidad con el estándar Matter facilitaría la interoperabilidad con un ecosistema más amplio de dispositivos domóticos y plataformas de hogar inteligente.
    
    \item \textbf{Migración a plataformas certificadas}: Transición del sistema operativo actual a Wind River Linux para facilitar los procesos de certificación como dispositivo sanitario y cumplimiento regulatorio avanzado.
\end{itemize}

Este trabajo constituye una contribución práctica al campo de las interfaces cerebro-computadora, proporcionando una base técnica sólida para desarrollos futuros en neurotecnología aplicada.
