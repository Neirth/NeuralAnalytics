% Chapter 1: Resumen
\chapter*{Resumen}
\addcontentsline{toc}{chapter}{Resumen}

Este proyecto presenta el desarrollo de un sistema de control domótico basado en interfaces cerebro-computadora (BCI). El objetivo principal es implementar un sistema no invasivo que permita la integración de señales cerebrales con dispositivos de iluminación inteligente, específicamente bombillas TP-Link Tapo.

El sistema utiliza la diadema Brainbit como dispositivo de adquisición de señales electroencefalográficas (EEG), procesando estas señales mediante una arquitectura de software que combina Deep Learning para procesamiento y una solución desarrollada en Rust para el consumo del modelo e interacción con el sistema de iluminación. 

El marco teórico aborda el desarrollo de un modelo de clasificación basado en señales EEG, explorando técnicas de procesamiento de señales y aprendizaje automático para la interpretación precisa de patrones cerebrales. Además, se profundiza en los requisitos técnicos y normativos necesarios para el desarrollo de dispositivos médicos, con especial énfasis en el estándar IEC 62304 para procesos del ciclo de vida del software de dispositivos médicos, y la implementación de sistemas operativos en tiempo real (RTOS) para garantizar la fiabilidad y seguridad del sistema.

La solución propuesta integra tecnologías modernas de procesamiento de señales cerebrales con sistemas de domótica, creando una interfaz natural e intuitiva para el control del entorno doméstico. Este proyecto representa un paso hacia la democratización de las interfaces cerebro-computadora en aplicaciones cotidianas.

\vspace{0.5cm}
\noindent\textbf{Palabras clave:} Interfaz cerebro-computadora, BCI, Domótica, Aprendizaje profundo, Rust, EEG, RTOS, Certificación médica

