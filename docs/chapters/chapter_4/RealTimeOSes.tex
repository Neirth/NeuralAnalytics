\chapter{Sistemas operativos en Tiempo Real}\label{ch:real_time_oses}

Los sistemas operativos en tiempo real (RTOS) \cite{Siewert_Pratt_2016} constituyen una rama especializada del software orientada a garantizar la previsibilidad temporal en entornos que requieren respuestas precisas. A diferencia de los sistemas operativos convencionales, estos sistemas priorizan la \textbf{predictibilidad temporal} sobre la velocidad de procesamiento, asegurando que cada operación se ejecute dentro de intervalos temporales específicos.

El concepto de computación en tiempo real emerge de la necesidad de procesar y responder a eventos del mundo físico con restricciones temporales bien definidas. En un RTOS, la corrección del sistema no solo depende de la exactitud lógica de los resultados, sino también del momento en que estos se producen. Esta dualidad en los requisitos (\textit{corrección lógica + corrección temporal}) distingue fundamentalmente a los RTOS de los sistemas operativos de propósito general.

La arquitectura de un RTOS se caracteriza por varios componentes esenciales:

\begin{itemize}
    \item \textbf{Planificador determinista}: Garantiza que las tareas críticas se ejecuten en momentos predecibles
    \item \textbf{Gestión de interrupciones}: Manejo prioritario de eventos externos con latencias acotadas
    \item \textbf{Gestión de memoria}: Esquemas de asignación y liberación que evitan indeterminismos temporales
\end{itemize}

En el ámbito de los sistemas embebidos, principal campo de aplicación de estos sistemas, el RTOS funciona como intermediario entre los componentes físicos y las operaciones de control. Un caso ilustrativo son los sistemas de seguridad en vehículos, donde cualquier retraso, incluso de microsegundos, podría resultar crítico. 

En algunos casos, estos requisitos de fiabilidad y predictibilidad temporal se extienden a sistemas de propósito general, especialmente en aplicaciones médicas y de consumo que demandan garantías temporales. 

Un ejemplo paradigmático son los dispositivos médicos implantables, donde la fiabilidad y predictibilidad temporal son requisitos fundamentales para garantizar la seguridad del paciente. Esta expansión ha llevado al desarrollo de estrictos marcos regulatorios y procesos de certificación específicos para cada sector de aplicación.

\newpage
\section{Taxonomía de Sistemas en Tiempo Real}
    La clasificación de los sistemas operativos en tiempo real se fundamenta principalmente en la criticidad de sus restricciones temporales. Esta taxonomía, establecida inicialmente por Liu y Layland en 1973 \cite{Siewert_Pratt_2016}, ha evolucionado para adaptarse a las necesidades modernas de la computación en tiempo real. La distinción fundamental se establece entre sistemas estrictos (\textit{hard real-time}) y flexibles (\textit{soft real-time}), aunque algunos autores reconocen una categoría intermedia denominada \textit{firm real-time}.

    \subsection{Sistemas de Tiempo Real Estricto}
        Los sistemas de tiempo real estricto (\textbf{hard real-time}) se caracterizan por la intolerancia absoluta a desviaciones temporales. En estos sistemas, el incumplimiento de un plazo temporal se considera un fallo catastrófico del sistema. La expresión matemática que define su comportamiento es:

        \begin{equation}
        \forall t \in T: R(t) \leq D(t)
        \end{equation}

        donde $R(t)$ representa el tiempo de respuesta y $D(t)$ el plazo temporal máximo permitido.

        Ejemplos paradigmáticos incluyen:
        \begin{itemize}
            \item \textbf{Sistemas de control nuclear}: Donde los tiempos de respuesta deben ser absolutamente predecibles para garantizar la seguridad
            \item \textbf{Mecanismos de frenado electrónico}: El ABS debe responder en microsegundos para prevenir accidentes
            \item \textbf{Sistemas quirúrgicos robotizados}: Requieren sincronización precisa para operaciones de alta precisión
        \end{itemize}

    Su implementación requiere sistemas de planificación \textbf{preemptivos} con prioridades estáticas, donde el tiempo máximo de ejecución (WCET) debe ser predecible y verificable. La planificación típicamente se basa en el algoritmo Rate Monotonic (RM) o Earliest Deadline First (EDF).

    \newpage
    \subsection{Sistemas de Tiempo Real Flexible}
        Los sistemas de tiempo real flexible (\textbf{soft real-time}) toleran cierta variabilidad en el cumplimiento de plazos temporales, operando bajo un modelo probabilístico donde:

        \begin{equation}
        P(R(t) \leq D(t)) \geq p_{min}
        \end{equation}

        siendo $p_{min}$ el nivel mínimo aceptable de cumplimiento temporal.

        Ejemplos representativos incluyen:
        \begin{itemize}
            \item \textbf{Plataformas de streaming multimedia}: Donde ocasionales pérdidas de frames son aceptables
            \item \textbf{Redes de monitorización industrial}: Con tolerancia a retrasos ocasionales en la actualización de datos
            \item \textbf{Sistemas de trading automatizado}: Donde el rendimiento promedio es más importante que garantías absolutas
        \end{itemize}

        Estos sistemas utilizan habitualmente planificadores basados en \textbf{tiempo compartido} con prioridades dinámicas, priorizando la optimización del rendimiento promedio sobre las garantías temporales absolutas. Las políticas de planificación suelen incluir variantes de Round Robin y planificación por prioridades dinámicas.

    \subsection{Consideraciones de Implementación}
        La elección entre implementaciones estrictas y flexibles debe considerar:
        \begin{itemize}
            \item \textbf{Análisis de Riesgos}: Evaluación de consecuencias por fallos temporales
            \item \textbf{Recursos Disponibles}: Capacidad de procesamiento y memoria
            \item \textbf{Costes}: Balance entre garantías temporales y complejidad del sistema
            \item \textbf{Certificación}: Requisitos regulatorios del dominio de aplicación
        \end{itemize}

\newpage
\section{Soluciones Comerciales para Hard Real-Time}

    \subsection{VxWorks (Wind River Systems)}
        VxWorks, desarrollado por Wind River Systems, representa el estándar industrial en sistemas embebidos críticos, especialmente en sectores como la aviónica, espacial y médico. Sus características principales incluyen:

        \subsubsection{Certificaciones y Cumplimiento Normativo}
            \begin{itemize}
                \item DO-178C Level A para sistemas aeroespaciales
                \item IEC 62304 para dispositivos médicos
                \item ISO 26262 ASIL D para automoción
            \end{itemize}
        \subsubsection{Características Técnicas}
            \begin{itemize}
                \item \textbf{Kernel}: Microkernel determinista con tiempos de interrupción $\le$ 50 ns
                \item \textbf{Memoria}: MMU con protección y aislamiento de espacios de memoria
                \item \textbf{Scheduling}: Planificador con 256 niveles de prioridad y herencia de prioridad
                \item \textbf{IPC}: Mecanismos de comunicación con latencia determinista
                \item \textbf{Multiprocesamiento}: Soporte para SMP y AMP con aislamiento de cores
            \end{itemize}

    \subsection{QNX Neutrino (BlackBerry)}
        QNX Neutrino, adquirido por BlackBerry, destaca por su arquitectura de microkernel distribuido y su alto nivel de fiabilidad:

        \subsubsection{Arquitectura}
            \begin{itemize}
                \item \textbf{Microkernel}: Núcleo de menos de 100KB
                \item \textbf{Servicios}: Arquitectura modular con servicios en espacio de usuario
                \item \textbf{IPC}: Sistema de mensajería síncrona y asíncrona con copy-on-write
                \item \textbf{Recuperación}: Capacidad de reinicio de componentes sin afectar al sistema
            \end{itemize}

        \subsubsection{Características Avanzadas}
            \begin{itemize}
                \item \textbf{Tiempo Real}: Garantías temporales con latencias $\le$ 100 $\mu$s
                \item \textbf{Seguridad}: Modelo de seguridad adaptativo con ASLR
                \item \textbf{Certificaciones}: IEC 61508 SIL3, IEC 62304 Clase C
            \end{itemize}

    \newpage
    \subsection{Zephyr RTOS (Linux Foundation)}
        Zephyr representa la alternativa open-source para sistemas embebidos críticos:

        \subsubsection{Diseño y Arquitectura}
            \begin{itemize}
                \item \textbf{Kernel}: Monolítico o microkernel configurable
                \item \textbf{Footprint}: Desde 8KB hasta configuraciones completas de 512KB
                \item \textbf{Scheduling}: Planificador configurable con hasta 32 niveles de prioridad
                \item \textbf{Certificación}: Proceso de certificación para IEC 61508 SIL 3/4
            \end{itemize}

        \subsubsection{Características Destacadas}
            \begin{itemize}
                \item \textbf{Drivers}: Más de 350 drivers para diferentes periféricos
                \item \textbf{Networking}: Soporte nativo para protocolos IoT (BLE, Thread, LoRaWAN)
                \item \textbf{Seguridad}: Subsistema de seguridad con aislamiento de memoria
                \item \textbf{Desarrollo}: Herramientas de desarrollo y depuración avanzadas
            \end{itemize}

\newpage
\section{Soluciones Comerciales para Soft Real-Time}
    \subsection{Wind River Linux (Wind River Systems)}
        Wind River Linux representa una solución empresarial certificada, basada en el Proyecto Yocto, específicamente diseñada para el desarrollo de sistemas embebidos que requieren garantías temporales flexibles:

        \subsubsection{Características Principales}
            \begin{itemize}
                \item \textbf{Base}: Kernel Linux 5.10 LTS con parche PREEMPT\_RT
                \item \textbf{Certificaciones}: ISO 9001:2015 y precertificación IEC 62304
                \item \textbf{Seguridad}: Monitorización continua de CVEs y mitigación
                \item \textbf{Cumplimiento}: Documentación SBOM y Open Chain 2.1
            \end{itemize}

        \subsubsection{Capacidades Industriales}
            \begin{itemize}
                \item \textbf{Soporte}: Mantenimiento garantizado de 5 años con extensión LTS
                \item \textbf{Actualizaciones}: Sistema OTA seguro mediante OSTree
                \item \textbf{Validación}: Más de 60.000 casos de prueba automatizados
                \item \textbf{Servicios}: Soporte profesional y consultoría disponible
            \end{itemize}

    \subsection{Poky Linux (Proyecto Yocto)}
        Poky constituye la distribución de referencia del Proyecto Yocto, proporcionando una base para el desarrollo de sistemas Linux embebidos con capacidades de tiempo real flexible:

        \subsubsection{Características Técnicas}
            \begin{itemize}
                \item \textbf{Kernel}: Linux con soporte opcional para PREEMPT\_RT
                \item \textbf{Tiempo Real}: Latencias configurables según necesidades
                \item \textbf{Optimización}: Control fino sobre el tamaño y rendimiento
                \item \textbf{Personalización}: Capacidad de eliminar componentes innecesarios
            \end{itemize}

        \subsubsection{Consideraciones de Desarrollo}
            \begin{itemize}
                \item \textbf{Mantenimiento}: Actualización manual de parches de seguridad
                \item \textbf{Soporte}: Basado en la comunidad, sin garantías comerciales
                \item \textbf{Certificación}: Requiere proceso de certificación propio
                \item \textbf{Validación}: Necesidad de desarrollar pruebas específicas
            \end{itemize}

\newpage
\section{Elección de RTOS para el Proyecto}
    La elección de Wind River Linux como sistema operativo para este proyecto se fundamenta en varios factores críticos:

    \subsection{Requisitos Temporales del Sistema}
        El proyecto requiere un sistema de tiempo real flexible (\textbf{soft real-time}), ya que:
        \begin{itemize}
            \item La detección de patrones cerebrales para la identificación de colores (rojo/verde) no requiere garantías temporales estrictas
            \item Las consecuencias de un retraso en la respuesta no comprometen la seguridad del usuario
            \item El control de iluminación mediante TP-Link Tapo tolera latencias moderadas
        \end{itemize}

    \subsection{Consideraciones Técnicas}
        Wind River Linux ofrece ventajas significativas para nuestro caso de uso:
        \begin{itemize}
            \item \textbf{Compatibilidad}: Garantiza el funcionamiento correcto del SDK de Brainflow
            \item \textbf{PREEMPT\_RT}: El parche de tiempo real proporciona las garantías temporales necesarias
            \item \textbf{Actualizaciones}: Sistema OTA que facilita el mantenimiento del software
        \end{itemize}

    \subsection{Aspectos Regulatorios}
        La precertificación IEC 62304 de Wind River Linux resulta crucial dado que:
        \begin{itemize}
            \item Reduce significativamente el esfuerzo de certificación del producto final
            \item Proporciona documentación regulatoria necesaria para el sector médico
            \item Garantiza el cumplimiento de estándares de seguridad y calidad
        \end{itemize}

    Esta combinación de factores hace que Wind River Linux sea la opción más adecuada para nuestro proyecto, proporcionando un equilibrio óptimo entre rendimiento, fiabilidad y cumplimiento normativo.
