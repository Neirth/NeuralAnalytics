\chapter{Implementación del Core}\label{ch:core_implementing}

Este capítulo documenta la implementación del núcleo de Neural Analytics, fase que comprende la traducción de conceptos arquitectónicos a código funcional. El procesamiento de señales biomédicas presenta complejidades técnicas inherentes que requieren consideraciones específicas de diseño. El capítulo describe sistemáticamente la construcción del núcleo del sistema, incluyendo la implementación de la arquitectura hexagonal, el patrón Model-View-Intent (MVI) y la integración con componentes externos.

La implementación de sistemas de procesamiento neurológico en tiempo real requiere desafíos técnicos específicos relacionados con la refactorización y optimización de componentes. La integración de bibliotecas externas y la gestión de concurrencia constituyen aspectos fundamentales para garantizar el funcionamiento correcto del sistema.

\section{Arquitectura Hexagonal}

La arquitectura hexagonal (\citeyear{martin2017clean})—también conocida como arquitectura de puertos y adaptadores—constituyó la base estructural de Neural Analytics. La implementación de esta arquitectura requiere una cuidadosa planificación y diseño para garantizar la separación de preocupaciones y la independencia tecnológica del núcleo de la aplicación.

Esta arquitectura implementa los principios establecidos por \citefullauthor{martin2017clean} sobre arquitectura limpia, fundamentados en la separación clara entre lógica de negocio e infraestructura externa. Estos principios proporcionan ventajas significativas en términos de mantenimiento, testing y escalabilidad, aunque su implementación práctica necesita un enfoque cuidadoso para evitar la complejidad innecesaria que pueden surgir a la hora de integrarlo con la capa de dominio.

\subsection{Estructura General}

La organización del paquete \texttt{neural\_analytics\_core} siguiendo los principios hexagonales requiere un proceso iterativo de refinamiento arquitectónico. La estructura se organiza de la siguiente manera:

\begin{itemize}
    \item \textbf{Dominio}: Contiene la lógica principal de la aplicación, diseñada para funcionar independientemente de tecnologías externas.
    \begin{itemize}
        \item \texttt{models}: Define las entidades y objetos de valor del dominio.
        \item \texttt{services}: Implementa servicios específicos del dominio como la inferencia del modelo.
        \item \texttt{ports}: Define las interfaces que conectan el dominio con el exterior.
        \item \texttt{use\_cases}: Implementa los casos de uso que coordinan el flujo de procesamiento.
        \item \texttt{state\_machine}: Maneja los estados de la aplicación mediante una máquina de estados finita.
    \end{itemize}
    
    \item \textbf{Infraestructura}: Contiene todos los adaptadores que implementan los puertos del dominio.
    \begin{itemize}
        \item \texttt{adapters/input}: Adaptadores para dispositivos de entrada como el EEG.
        \item \texttt{adapters/output}: Adaptadores para dispositivos de salida como las bombillas inteligentes.
    \end{itemize}
\end{itemize}

\subsection{Puertos y Adaptadores}

Los puertos definen interfaces abstractas que el dominio utiliza para comunicarse con el exterior, mientras que los adaptadores proporcionan implementaciones concretas de estas interfaces. Esta separación permite la independencia tecnológica del núcleo de la aplicación.

La arquitectura implementada permite la refactorización de componentes para lograr la separación adecuada de responsabilidades, manteniendo la funcionalidad del sistema.

\subsubsection{Puertos de Entrada}

El puerto principal de entrada define las operaciones necesarias para la comunicación con dispositivos de electroencefalografía. Esta interfaz establece un contrato que garantiza la compatibilidad con sistemas concurrentes y la gestión segura de recursos compartidos.

Las funcionalidades principales incluyen:

\begin{itemize}
    \item \textbf{Gestión de conectividad}: Operaciones para establecer, verificar y finalizar la comunicación con el dispositivo EEG.
    \item \textbf{Adquisición de datos}: Extracción de valores de impedancia para evaluación de calidad de señal, y captura de señales electroencefalográficas de las regiones temporales y occipitales del cerebro.
    \item \textbf{Control de configuración}: Gestión de los modos operativos del dispositivo según los requisitos del proceso de adquisición.
\end{itemize}

La arquitectura del puerto utiliza estructuras de datos optimizadas para el procesamiento eficiente de señales biomédicas, facilitando la integración con algoritmos de análisis posterior.

\subsubsection{Puertos de Salida}

El puerto principal de salida define las operaciones para el control de dispositivos domóticos inteligentes. Esta interfaz soporta operaciones asíncronas, característica esencial para la comunicación de red no bloqueante en sistemas de tiempo real.

Las características principales comprenden:

\begin{itemize}
    \item \textbf{Control de estado}: Gestión de estados específicos de los dispositivos, incluyendo activación con configuraciones de color basadas en la detección neurológica.
    \item \textbf{Procesamiento asíncrono}: Las operaciones se ejecutan en contextos no bloqueantes, permitiendo el manejo eficiente de errores de comunicación de red.
    \item \textbf{Arquitectura concurrente}: El diseño garantiza el uso seguro en entornos de ejecución multihilo.
\end{itemize}

La abstracción implementada encapsula los detalles específicos de comunicación con diferentes fabricantes de dispositivos domóticos, proporcionando extensibilidad y mantenibilidad al sistema.

\subsubsection{Adaptadores}

Los adaptadores implementan los puertos para tecnologías específicas, concentrando la complejidad técnica de integración con bibliotecas externas:

\begin{itemize}
    \item \texttt{BrainFlowAdapter}: Implementa \texttt{EegHeadsetPort} usando la biblioteca BrainFlow para comunicarse con el dispositivo BrainBit. La implementación de este adaptador requiere consideraciones específicas para garantizar un funcionamiento estable y fiable del sistema.
    \item \texttt{TapoSmartBulbAdapter}: Implementa \texttt{SmartBulbPort} para controlar bombillas inteligentes Tapo. Este adaptador presenta una implementación más directa debido a la naturaleza de la API del fabricante.
\end{itemize}

\section{Consumo del SDK de BrainFlow}

BrainFlow constituye una biblioteca de código abierto que proporciona una API unificada para dispositivos de neurointerfaz (BCI), facilitando la adquisición, procesamiento y visualización de datos cerebrales. Neural Analytics utiliza BrainFlow para establecer comunicación con el dispositivo BrainBit. La integración de esta biblioteca representa un componente fundamental en la arquitectura del sistema, proporcionando la abstracción necesaria para el manejo de hardware especializado.

La selección de BrainFlow se fundamentó en criterios como la madurez del proyecto, su compatibilidad multiplataforma y el soporte para diversos dispositivos de electroencefalografía. Esta biblioteca ofrece una interfaz estandarizada que abstrae las especificidades de cada fabricante de hardware.

\subsection{Inicialización y Configuración}

El adaptador correspondiente inicializa el dispositivo BrainBit mediante la API de BrainFlow. El proceso de configuración comprende los siguientes componentes:

\begin{enumerate}
    \item \textbf{Configuración de parámetros}: Establecimiento de la dirección MAC del dispositivo BrainBit y definición de tiempos de espera apropiados para garantizar la estabilidad de la conexión.
    \item \textbf{Selección de identificador}: Especificación del identificador de placa correspondiente al modelo BrainBit según las especificaciones del fabricante.
    \item \textbf{Instanciación del gestor}: Creación del gestor de comunicación que coordina todas las operaciones con el dispositivo.
\end{enumerate}

Este proceso establece un canal de comunicación bidireccional con el dispositivo EEG, permitiendo tanto la configuración como la recepción de datos en tiempo real. La arquitectura implementada proporciona robustez en el manejo de conexiones y tolerancia a fallos de comunicación.

\subsection{Adquisición de Datos}

La adquisición de datos electroencefalográficos constituye un proceso fundamental que requiere precisión en el manejo de señales biomédicas. El sistema implementa una metodología estructurada para la extracción de datos de cuatro canales específicos (T3, T4, O1 y O2):

\begin{enumerate}
    \item \textbf{Validación de conectividad}: Verificación del estado del dispositivo antes de proceder con la adquisición de datos.
    
    \item \textbf{Captura de muestras}: Solicitud de un búfer conteniendo las últimas 256 muestras de todos los canales disponibles.
    
    \item \textbf{Selección de canales}: Extracción específica de los canales correspondientes a las regiones temporales y occipitales del cerebro, relevantes para la detección de patrones relacionados con estímulos visuales.
    
    \item \textbf{Estructuración de datos}: Organización de los datos en estructuras optimizadas donde cada canal se asocia con su correspondiente serie temporal.
\end{enumerate}

Esta metodología permite la obtención de señales EEG de las regiones cerebrales específicamente involucradas en el procesamiento de información visual, facilitando la posterior detección de patrones asociados a la ideación de colores.

\section{Patrón Model-View-Intent (MVI)}

Neural Analytics adopta principios del patrón Model-View-Intent (MVI) adaptados a una arquitectura basada en eventos para gestionar la comunicación entre la interfaz de usuario y el núcleo de la aplicación. Aunque este patrón arquitectónico se encuentra consolidado primordialmente en el desarrollo Android tal como establece \cite{dumbravan2022clean}, y que se fundamenta en principios de programación funcional reactiva que \cite{blackheath2016functional} describe exhaustivamente, la implementación en este proyecto incorpora dichos principios de manera híbrida, combinándolos con un sistema de eventos asíncronos específicamente diseñado para el procesamiento de señales biomédicas en tiempo real.

La arquitectura implementada proporciona ventajas específicas para sistemas de procesamiento neurológico como Neural Analytics. El flujo unidireccional de eventos desde el núcleo hacia la interfaz garantiza la trazabilidad de estados durante el procesamiento de señales EEG, aspecto fundamental para el análisis de patrones cerebrales. Además, la separación arquitectónica entre el contexto de aplicación (modelo) y la interfaz gráfica permite que las operaciones críticas de inferencia neurológica se ejecuten de forma asíncrona e independiente, optimizando el rendimiento del sistema sin bloquear la experiencia de usuario.

\subsection{Componentes del Patrón MVI}

La organización de los componentes según los principios del patrón MVI establece una arquitectura coherente y funcional:

\begin{itemize}
    \item \textbf{Model}: Representado por el contexto de la aplicación, que centraliza todo el estado del sistema en una ubicación única. Esta aproximación facilita la consistencia de datos y simplifica la gestión de estados complejos.
    \item \textbf{View}: Desarrollada mediante tecnologías de interfaz declarativa, proporcionando una experiencia de usuario fluida y responsiva para la visualización de datos neurológicos.
    \item \textbf{Intent}: Materializado a través de comandos que se procesan mediante un bus de comandos, donde cada comando ejecuta cambios específicos en el estado del sistema. La granularidad de los comandos se define para mantener un equilibrio entre funcionalidad y complejidad arquitectónica.
\end{itemize}

\subsection{Flujo de Datos}

La implementación del flujo de datos según el patrón MVI establece una secuencia ordenada de procesamiento:

\begin{enumerate}
    \item \textbf{Intención del usuario}: El usuario interactúa con la interfaz gráfica mediante acciones como selecciones de botones o modificaciones en configuraciones.
    \item \textbf{Generación de comandos}: La interfaz traduce las acciones del usuario en comandos específicos que se transmiten al núcleo del sistema. Cada comando se diseña con semántica explícita para garantizar comportamientos predecibles.
    \item \textbf{Procesamiento de casos de uso}: Un caso de uso específico procesa el comando recibido, aplicando la lógica de validación correspondiente para gestionar adecuadamente las acciones del usuario.
    \item \textbf{Actualización del modelo}: El caso de uso modifica el estado del contexto según la lógica de negocio aplicable.
    \item \textbf{Emisión de eventos}: Los cambios en el estado generan eventos automáticamente. El mecanismo implementado optimiza la generación de eventos para evitar saturación del sistema.
    \item \textbf{Actualización de la vista}: Los eventos son procesados por el manejador correspondiente en la interfaz gráfica, que actualiza la visualización. La sincronización entre hilos se gestiona mediante primitivas apropiadas de concurrencia.
\end{enumerate}

\subsection{Manejador de Eventos}

La implementación del manejador de eventos procesa los eventos del núcleo y actualiza la interfaz gráfica mediante un mecanismo estructurado:

\begin{enumerate}
    \item \textbf{Recepción del evento}: El manejador recibe el identificador del evento y los datos asociados (impedancias, datos del dispositivo, color detectado). El procesamiento de estos datos requiere parseo específico según el tipo de evento.
    
    \item \textbf{Sincronización con el hilo de la interfaz}: Dado que los eventos se generan en hilos diferentes al de la interfaz de usuario, se implementa un mecanismo de sincronización segura para las modificaciones de la interfaz.
    
    \item \textbf{Gestión de referencias}: El sistema obtiene referencias a la ventana principal mediante referencias débiles, evitando ciclos de referencia y fugas de memoria.
    
    \item \textbf{Procesamiento condicional}: Según el tipo de evento recibido, se ejecutan acciones específicas:
    \begin{itemize}
        \item Los eventos de inicialización activan la vista de bienvenida del sistema.
        \item Los eventos de conexión del dispositivo EEG activan la transición a la vista de calibración, incluyendo validaciones de estabilidad de conexión.
        \item Otros eventos gestionan transiciones de vista o actualizaciones de datos específicas según la lógica de la aplicación.
    \end{itemize}
\end{enumerate}

Este diseño logra una separación clara entre la lógica de negocio y la presentación, manteniendo la coherencia arquitectónica del patrón MVI implementado.

\section{Interconexión con el sistema domótico}

La integración de Neural Analytics con dispositivos domóticos permite que las intenciones del usuario se materialicen en el entorno físico. La implementación actual utiliza bombillas inteligentes Tapo, seleccionadas por su disponibilidad comercial y compatibilidad técnica.

La arquitectura se diseñó considerando la futura integración de otros sistemas domóticos, proporcionando extensibilidad mediante la abstracción de puertos y adaptadores.

\subsection{Adaptador para Bombillas Inteligentes}

El adaptador para bombillas inteligentes implementa el puerto correspondiente para el control de dispositivos domóticos. La implementación aborda varios aspectos técnicos fundamentales:

\begin{itemize}
    \item \textbf{Gestión de estado interno}: El adaptador mantiene referencias al cliente de comunicación con dispositivos Tapo, instancias específicas del modelo de bombilla e indicadores del estado de conexión.
    
    \item \textbf{Establecimiento de conexión}: La conexión se realiza mediante bibliotecas especializadas que gestionan la autenticación y comunicación segura con la bombilla inteligente, configurando las credenciales necesarias y estableciendo sesiones persistentes.
    
    \item \textbf{Control de color}: La implementación traduce los conceptos de alto nivel del dominio a comandos específicos de la API:
    \begin{itemize}
        \item Para detección de color rojo: configuración con parámetros de color rojo de alta intensidad visual.
        \item Para detección de color verde: configuración con parámetros de color verde de intensidad moderada.
        \item Para estados neutros: configuración de estado apagado o neutro del dispositivo.
    \end{itemize}
    
    \item \textbf{Gestión de errores}: Se implementa un sistema robusto para el manejo de excepciones y errores de red, incluyendo configuración apropiada de timeouts y reintentos automáticos.
\end{itemize}

Esta abstracción permite que el sistema principal opere con conceptos de alto nivel, encapsulando los detalles específicos de comunicación con los dispositivos domóticos.

\subsection{Integración con la Máquina de Estados}

La integración del sistema domótico en el flujo de la aplicación se realiza mediante casos de uso específicos que operan sobre el contexto de la aplicación. El caso de uso para actualizar el estado de la bombilla opera mediante el siguiente proceso:

\begin{enumerate}
    \item \textbf{Recepción del contexto y comando}: El caso de uso recibe acceso al contexto global de la aplicación y el comando específico para actualizar el estado de la bombilla.
    
    \item \textbf{Extracción del color detectado}: Se consulta el contexto para determinar el último color identificado en el pensamiento del usuario, incluyendo validaciones para asegurar la existencia de datos válidos.
    
    \item \textbf{Adquisición de acceso exclusivo}: Mediante mecanismos de bloqueo de escritura, se obtiene acceso exclusivo al adaptador de la bombilla inteligente para evitar condiciones de carrera.
    
    \item \textbf{Actualización del estado}: Se invoca el método correspondiente para cambiar el color de la bombilla, utilizando el color detectado como parámetro.
    
    \item \textbf{Gestión de errores}: Se implementa un sistema de propagación de errores para notificar adecuadamente cualquier fallo durante el proceso de comunicación con dispositivos de red.
\end{enumerate}

Esta integración permite que los cambios en la detección del pensamiento del usuario se reflejen en el entorno físico, estableciendo el bucle cerrado de interacción entre el cerebro y el entorno objetivo del proyecto.

\newpage
\subsection{Preparación para una futura integración con Matter}

El proyecto utiliza la arquitectura hexagonal considerando la futura integración de Matter, estándar de conectividad para IoT que proporciona interoperabilidad entre dispositivos de diferentes fabricantes.

Para soportar Matter en versiones futuras, será necesario:

\begin{enumerate}
    \item Implementar un nuevo adaptador que cumpla con el puerto correspondiente utilizando la API de Matter, aprovechando la abstracción existente.
    \item Registrar el nuevo adaptador en el sistema de inyección de dependencias. Esta parte ya está preparada en la arquitectura actual.
    \item Configurar la aplicación para usar el adaptador de Matter en lugar del adaptador actual mediante cambios en la configuración del sistema.
\end{enumerate}

Este enfoque ilustra cómo la arquitectura hexagonal permite extender el sistema con nuevas tecnologías sin modificar la lógica de negocio central, cumpliendo el objetivo de flexibilidad y extensibilidad establecido durante la fase de diseño.

\section{Implementación de la interfaz gráfica}

El desarrollo de la interfaz gráfica de Neural Analytics utiliza el framework Slint, seleccionado tras evaluar diversas opciones disponibles. Slint proporciona interfaces gráficas declarativas y eficientes para aplicaciones desarrolladas en Rust.

\subsection{Estructura de la GUI}

La organización de la interfaz gráfica se estructura mediante los siguientes componentes:

\begin{itemize}
    \item \textbf{Vistas principales}: Pantallas correspondientes a los diferentes estados del sistema, diseñadas priorizando la simplicidad y claridad funcional.
    \item \textbf{Componentes reutilizables}: Elementos de interfaz como botones, etiquetas y visualizaciones que optimizan el desarrollo mediante reutilización.
    \item \textbf{Manejadores de eventos}: Funciones que procesan acciones del usuario y eventos del sistema, implementando sincronización apropiada para evitar conflictos entre hilos.
\end{itemize}

\subsection{Integración con el Core}

La comunicación entre la interfaz gráfica y el núcleo se establece mediante un manejador de eventos estructurado. El proceso implementado comprende:

\begin{enumerate}
    \item \textbf{Inicialización de la interfaz gráfica}: Se crea la ventana principal de la aplicación utilizando las capacidades declarativas de Slint.
    
    \item \textbf{Gestión de referencias}: Se almacenan referencias débiles a la ventana principal en estructuras protegidas por mutex, evitando ciclos de referencia mientras se permite acceso desde callbacks asíncronos.
    
    \item \textbf{Configuración de manejadores de eventos}: Se establecen los callbacks para responder a las acciones del usuario:
    \begin{itemize}
        \item Inicialización del proceso principal mediante hilos asíncronos que activan el núcleo del sistema.
        \item Configuración de manejadores de eventos que permiten la comunicación bidireccional entre núcleo e interfaz.
    \end{itemize}
    
    \item \textbf{Ejecución del bucle de eventos}: Se inicia el bucle principal de la interfaz gráfica, procesando interacciones del usuario y actualizando la visualización según los eventos recibidos.
\end{enumerate}

Este diseño permite la operación asíncrona de la interfaz gráfica y el núcleo, aprovechando múltiples hilos para tareas intensivas como el procesamiento de señales EEG, manteniendo la responsividad de la interfaz de usuario.

\newpage
\section{Conclusiones y Justificación de Decisiones Arquitectónicas}

La implementación del núcleo de Neural Analytics con arquitectura hexagonal demuestra que las decisiones arquitectónicas adoptadas proporcionan ventajas significativas para sistemas de procesamiento de señales EEG orientados a aplicaciones médicas.

\subsection{Alineación con los Requisitos del Proyecto}

La arquitectura implementada cumple directamente con varios requisitos fundamentales establecidos en las fases iniciales del proyecto:

\begin{itemize}
    \item \textbf{Portabilidad}: La interfaz de puertos permite intercambiar dispositivos EEG sin modificar el núcleo del sistema, facilitando las pruebas en diferentes plataformas como Raspberry Pi.
    
    \item \textbf{Interoperabilidad}: El diseño facilita la integración con diferentes sistemas domóticos a través del puerto correspondiente, preparando la adopción futura del estándar Matter.
    
    \item \textbf{Cumplimiento normativo}: La separación clara de componentes facilita la validación y verificación según la norma UNE-EN 62304, requisito fundamental para software de dispositivos médicos.
    
    \item \textbf{Procesamiento en tiempo real}: La arquitectura permite que los componentes críticos de procesamiento de señales funcionen en hilos independientes de la interfaz de usuario, garantizando el rendimiento del sistema.
\end{itemize}

\subsection{Beneficios de la Arquitectura Implementada}

La arquitectura implementada proporciona ventajas prácticas significativas:

\begin{itemize}
    \item \textbf{Desarrollo modular}: La arquitectura facilita el trabajo independiente en interfaz gráfica y adaptadores de hardware sin interferencias entre componentes.
    
    \item \textbf{Pruebas simplificadas}: La implementación de adaptadores simulados permite probar el sistema completo sin dependencia del hardware físico, optimizando los ciclos de desarrollo.
    
    \item \textbf{Evolución tecnológica}: Las actualizaciones de bibliotecas externas requieren únicamente modificaciones en los adaptadores correspondientes, sin afectar el resto del sistema.
    
    \item \textbf{Trazabilidad}: La estructura facilita el seguimiento del cumplimiento de requisitos durante las revisiones de calidad del software.
\end{itemize}

\subsection{Impacto en la Calidad del Software}

La arquitectura implementada proporciona un impacto positivo verificable en la calidad del sistema:

\begin{itemize}
    \item \textbf{Fiabilidad}: El sistema demuestra comportamiento predecible ante desconexiones de hardware y valida la calidad de señales EEG mediante datos de impedancia antes de generar predicciones.
    
    \item \textbf{Mantenibilidad}: El patrón MVI y la máquina de estados proporcionan un flujo de datos unidireccional que facilita la depuración y el mantenimiento del código.
    
    \item \textbf{Extensibilidad}: La capacidad de actualizar modelos de inferencia sin modificar el código de la aplicación permite el despliegue de nuevos modelos de detección de patrones cerebrales.
    
    \item \textbf{Portabilidad}: El sistema funciona adecuadamente en macOS y Linux sin adaptaciones significativas en la lógica de negocio.
\end{itemize}

La implementación basada en arquitectura hexagonal proporciona una base sólida que cumple con los requisitos actuales del proyecto Neural Analytics y permite su evolución futura de manera sostenible, alineada con estándares médicos y tecnológicos emergentes.
