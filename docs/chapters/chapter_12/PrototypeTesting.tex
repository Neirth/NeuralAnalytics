\chapter{Validación del Prototipo}\label{ch:prototype_testing}

Este capítulo documenta el proceso de validación del prototipo Neural Analytics de acuerdo con la normativa aplicable para software de dispositivos médicos. Se detallan las estrategias de prueba implementadas, los resultados obtenidos y su adecuación a los estándares requeridos.

\section{Marco Normativo de Validación}

\subsection{Normativa Aplicable}

La validación del software Neural Analytics se ha realizado siguiendo las directrices establecidas en la norma UNE-EN 62304:2007/A1:2016, "Software de dispositivos médicos - Procesos del ciclo de vida del software". Esta norma establece los requisitos para los procesos del ciclo de vida del desarrollo y mantenimiento del software en dispositivos médicos, y es armonizada con la Directiva Europea de Productos Sanitarios 93/42/CEE.

\subsection{Clasificación del Software}

De acuerdo con la sección 4.3 de la norma UNE-EN 62304, el software Neural Analytics ha sido clasificado como dispositivo de \textbf{Clase A}, dado que:

\begin{itemize}
    \item No puede contribuir directamente a situaciones peligrosas, ya que funciona únicamente como herramienta de monitorización sin capacidad de efectuar acciones directas sobre el paciente.
    \item Su uso está destinado a aplicaciones no críticas de interfaz cerebro-computadora.
    \item El sistema incluye restricciones de uso explícitas que previenen su aplicación en escenarios clínicos críticos.
\end{itemize}

Esta clasificación determina el nivel de rigor aplicado a las pruebas y documentación del software, según lo estipulado en la norma.

\section{Estrategia de Pruebas}

Siguiendo los requisitos de la norma UNE-EN 62304 para software de Clase A, se ha implementado una estrategia de pruebas estructurada en tres niveles:

\begin{itemize}
    \item \textbf{Pruebas unitarias}: Verificación de componentes individuales del software
    \item \textbf{Pruebas de integración}: Verificación de la interacción entre componentes
    \item \textbf{Pruebas del sistema}: Verificación del funcionamiento global del sistema
\end{itemize}

Como parte del alcance definido para este proyecto, se ha focalizado la estrategia de pruebas en los dos componentes principales: \texttt{neural\_analytics\_core} y \texttt{neural\_analytics\_gui}, al ser estos los módulos que constituyen el producto final destinado al usuario.

\subsection{Herramientas y Entorno de Pruebas}

Para la ejecución de las pruebas se ha utilizado el siguiente entorno:

\begin{table}[ht]
    \centering
    \begin{tabular}{|l|l|}
        \hline
        \textbf{Elemento} & \textbf{Descripción} \\
        \hline
        Hardware principal & Raspberry Pi 4 Model B (8GB RAM) \\
        \hline
        Sistema operativo & Poky Linux 64-bit \\
        \hline
        Dispositivo EEG & BrainBit (4 canales) \\
        \hline
        Dispositivos actuadores & Bombillas inteligentes Tapo L530E \\
        \hline
        Framework de pruebas unitarias & Rust Test Framework \\
        \hline
        Herramientas de cobertura & grcov, cargo-llvm-cov \\
        \hline
        Herramientas de monitorización & Raspberry Pi Diagnostics, htop \\
        \hline
    \end{tabular}
    \caption{Entorno de pruebas para Neural Analytics}
    \label{tab:test_environment}
\end{table}

\section{Pruebas Unitarias}

\subsection{Estrategia de Pruebas Unitarias}

Las pruebas unitarias se han diseñado para verificar el correcto funcionamiento de los componentes individuales del software, con especial énfasis en:

\begin{itemize}
    \item Funcionamiento correcto de módulos aislados
    \item Manejo adecuado de casos límite
    \item Gestión de errores
    \item Consistencia en la interfaz de los componentes
\end{itemize}

De acuerdo con la sección 5.5.3 de la norma UNE-EN 62304, para cada prueba unitaria se han definido criterios de aceptación explícitos antes de su ejecución.

\newpage
\subsection{Pruebas Unitarias del Core (\texttt{neural\_analytics\_core})}

El módulo core del sistema, responsable de la lógica central de procesamiento y gestión de eventos, ha sido sometido a pruebas unitarias exhaustivas:

\begin{table}[ht]
    \centering
    \begin{tabular}{|p{2.5cm}|p{5cm}|p{3cm}|p{2cm}|}
        \hline
        \textbf{Módulo} & \textbf{Aspectos probados} & \textbf{Criterio de aceptación} & \textbf{Resultado} \\
        \hline
        Adaptadores EEG & Conexión, lectura de datos, gestión de desconexiones & Lectura consistente de datos sin pérdida de muestras & PASA \\
        \hline
        Servicio de inferencia & Carga del modelo ONNX, preprocesamiento, inferencia & Predicciones coherentes con valores esperados & PASA \\
        \hline
        Controlador de dispositivos & Conexión con bombillas inteligentes, cambio de estados & Respuesta en < 300ms a cambios de estado & PASA \\
        \hline
        Máquina de estados & Transiciones correctas entre estados del sistema & Comportamiento consistente ante eventos & PASA \\
        \hline
        Comunicación entre componentes & Bus de eventos, suscripciones, publicaciones & Entrega confiable de eventos & PASA \\
        \hline
    \end{tabular}
    \caption{Resultados de pruebas unitarias del módulo Core}
    \label{tab:unit_tests_core}
    
A continuación, se detallan algunas de las pruebas unitarias implementadas para los componentes clave del \texttt{neural\_analytics\_core}:

\subsubsection{Pruebas del Servicio de Inferencia}

El servicio de inferencia, responsable de la clasificación de patrones cerebrales mediante el modelo ONNX, ha sido probado a través de las siguientes pruebas unitarias:

\begin{itemize}
    \item \textbf{test\_model\_loading}: Verifica que el modelo ONNX se carga correctamente desde diferentes rutas de archivo, incluyendo casos de archivos inexistentes.
    
    \item \textbf{test\_prediction\_with\_valid\_data}: Comprueba que el servicio realiza predicciones coherentes cuando recibe datos EEG válidos, comparando las salidas contra valores de referencia conocidos.
    
    \item \textbf{test\_prediction\_with\_invalid\_data}: Verifica el comportamiento del servicio ante datos mal formateados o fuera de rango, asegurando que maneja adecuadamente los errores.
    
    \item \textbf{test\_prediction\_performance}: Mide el tiempo de inferencia para garantizar que las predicciones se realizan dentro del umbral aceptable (< 100ms por predicción).
\end{itemize}

\subsubsection{Pruebas de los Adaptadores de Entrada}

Los adaptadores responsables de la comunicación con el dispositivo BrainBit han sido probados mediante:

\begin{itemize}
    \item \textbf{test\_headset\_connection}: Verifica el ciclo completo de conexión con el dispositivo BrainBit, comprobando la detección de dispositivos y establecimiento de conexión.
    
    \item \textbf{test\_headset\_data\_capture}: Asegura que los datos EEG se capturan correctamente de todos los canales disponibles (T3, T4, O1, O2) con las frecuencias de muestreo esperadas.
    
    \item \textbf{test\_impedance\_check}: Comprueba el funcionamiento del mecanismo de verificación de impedancias para todos los electrodos, validando que los valores están dentro de los rangos aceptables.
    
    \item \textbf{test\_disconnection\_handling}: Verifica la gestión adecuada de desconexiones imprevistas, asegurando la liberación de recursos y la posibilidad de reconexión.
\end{itemize}

\subsubsection{Pruebas de la Máquina de Estados}

La máquina de estados, componente central que gestiona las transiciones entre los diferentes modos de operación, ha sido probada mediante:

\begin{itemize}
    \item \textbf{test\_initial\_state}: Verifica que la máquina de estados se inicializa en el estado correcto.
    
    \item \textbf{test\_state\_transitions}: Comprueba todas las transiciones válidas entre estados, asegurando que ocurren correctamente en respuesta a los eventos apropiados.
    
    \item \textbf{test\_invalid\_transitions}: Verifica que las transiciones no permitidas son rechazadas adecuadamente, manteniendo la consistencia del sistema.
    
    \item \textbf{test\_event\_handling}: Asegura que los eventos generados por diferentes componentes son procesados correctamente por la máquina de estados.
\end{itemize}
\end{table}

\subsubsection{Prueba destacada: Servicio de Inferencia}

Una prueba particularmente relevante fue la verificación del servicio de inferencia, el cual recibe datos EEG preprocesados y debe producir predicciones precisas sobre la intención del usuario. Esta prueba verifica no solo el procesamiento técnico del modelo ONNX, sino también la precisión semántica de sus predicciones, asegurando que el núcleo de la aplicación (la clasificación de ondas cerebrales) funcione correctamente.

Para la prueba se utilizaron datos EEG simulados que representan diferentes patrones cerebrales, verificando que:
\begin{itemize}
    \item El servicio cargara correctamente el modelo ONNX
    \item Los datos de entrada fueran preprocesados adecuadamente
    \item Las predicciones coincidieran con las categorías esperadas
    \item El nivel de confianza de las predicciones superara el umbral establecido (70\%)
\end{itemize}

\newpage
\subsection{Pruebas Unitarias de la Interfaz (\texttt{neural\_analytics\_gui})}

Las pruebas unitarias para el componente de interfaz gráfica se centraron en:

\begin{table}[ht]
    \centering
    \begin{tabular}{|p{2.5cm}|p{5cm}|p{3cm}|p{2cm}|}
        \hline
        \textbf{Componente} & \textbf{Aspectos probados} & \textbf{Criterio de aceptación} & \textbf{Resultado} \\
        \hline
        Vista principal & Renderizado de elementos, navegación entre secciones & Visualización correcta y respuesta a interacciones & PASA \\
        \hline
        Visualización de señales & Representación gráfica de señales EEG en tiempo real & Actualización fluida (>25 FPS) & PASA \\
        \hline
        Módulo de calibración & Detección de impedancias, guía de usuario & Feedback preciso sobre calidad de contacto & PASA \\
        \hline
        Panel de configuración & Gestión de preferencias, validación de entradas & Persistencia de configuraciones & PASA \\
        \hline
        Indicadores de estado & Visualización del estado del sistema y predicciones & Correspondencia con estados internos & PASA \\
        \hline
    \end{tabular}
    \caption{Resultados de pruebas unitarias del módulo GUI}
    \label{tab:unit_tests_gui}
\end{table}

\newpage
\subsubsection{Cobertura de Código}

Se ha utilizado \texttt{cargo-llvm-cov} para medir la cobertura de código de las pruebas unitarias:

\begin{table}[ht]
    \centering
    \begin{tabular}{|l|c|}
        \hline
        \textbf{Módulo} & \textbf{Cobertura de código} \\
        \hline
        neural\_analytics\_core & 87.3\% \\
        \hline
        neural\_analytics\_gui & 82.1\% \\
        \hline
    \end{tabular}
    \caption{Cobertura de pruebas unitarias}
    \label{tab:code_coverage}
\end{table}


\section{Pruebas de Integración}

\subsection{Enfoque de Pruebas de Integración}

Las pruebas de integración se han enfocado en verificar la correcta interacción entre los distintos componentes del sistema, siguiendo la sección 5.6 de la norma UNE-EN 62304. Se han realizado pruebas en dos niveles:

\begin{enumerate}
    \item \textbf{Integración intra-módulo}: Verificando la correcta interacción entre componentes del mismo módulo
    \item \textbf{Integración inter-módulo}: Verificando la interacción entre el módulo core y la interfaz gráfica
\end{enumerate}

\subsection{Resultados de Pruebas de Integración}

Las pruebas de integración realizadas han arrojado los siguientes resultados:

\begin{table}[ht]
    \centering
    \begin{tabular}{|p{4cm}|p{7cm}|p{2cm}|}
        \hline
        \textbf{Escenario de integración} & \textbf{Descripción} & \textbf{Resultado} \\
        \hline
        Core $\leftrightarrow$ Adaptador EEG & Flujo de datos desde el dispositivo EEG al procesador del Core & PASA \\
        \hline
        Core $\leftrightarrow$ Controlador de dispositivos & Activación de bombillas inteligentes tras la detección de intenciones & PASA \\
        \hline
        Core $\leftrightarrow$ Interfaz & Visualización de estados del sistema y datos en tiempo real & PASA \\
        \hline
        Servicio de inferencia $\leftrightarrow$ Máquina de estados & Transición correcta de estados basada en resultados de inferencia & PASA \\
        \hline
        Módulo de configuración $\leftrightarrow$ Componentes del sistema & Aplicación efectiva de configuraciones de usuario & PASA \\
        \hline
    \end{tabular}
    \caption{Resultados de pruebas de integración}
    \label{tab:integration_tests}
\end{table}

\subsubsection{Prueba de Integración Destacada: Flujo Completo de Procesamiento}

Una de las pruebas de integración más significativas verificó el flujo completo de procesamiento del sistema. Esta prueba evalúa la cadena completa de procesamiento desde la entrada de datos hasta la actuación de los dispositivos.

El procedimiento seguido fue:
\begin{enumerate}
    \item Configurar el sistema completo con adaptadores simulados
    \item Inyectar datos EEG que representan el patrón cerebral asociado al color "verde"
    \item Ejecutar un ciclo completo de procesamiento
    \item Verificar que el estado de la bombilla inteligente cambia al color verde
    \item Comprobar que la interfaz de usuario se actualiza mostrando la predicción correcta
    \item Confirmar que el nivel de confianza supera el umbral mínimo establecido (75\%)
\end{enumerate}

Esta prueba verifica la integración end-to-end del sistema, desde la recepción de datos EEG hasta la activación de dispositivos y actualización de la interfaz, confirmando que todos los componentes trabajan juntos correctamente.

\section{Pruebas del Sistema}

Las pruebas del sistema se han diseñado para evaluar el comportamiento del sistema completo en condiciones reales y similares a las de uso final, conforme a la sección 5.7 de la norma UNE-EN 62304.

\newpage
\subsection{Casos de Prueba del Sistema}

Se han diseñado y ejecutado los siguientes casos de prueba del sistema:

\begin{table}[ht]
    \centering
    \begin{tabular}{|p{1cm}|p{3.5cm}|p{3.5cm}|p{3.5cm}|p{1.5cm}|}
        \hline
        \textbf{ID} & \textbf{Descripción} & \textbf{Procedimiento} & \textbf{Criterio de aceptación} & \textbf{Resultado} \\
        \hline
        SYS-01 & Inicialización del sistema & Iniciar la aplicación y verificar la carga de todos los módulos & Sistema operativo en <10s sin errores & PASA \\
        \hline
        SYS-02 & Conexión con dispositivo EEG & Encender el dispositivo BrainBit y conectarlo con la aplicación & Conexión establecida y datos fluyendo & PASA \\
        \hline
        SYS-03 & Calibración del dispositivo & Seguir el procedimiento de calibración & Impedancias aceptables en todos los canales & PASA \\
        \hline
        SYS-04 & Detección de pensamiento "rojo" & Usuario piensa en color rojo durante 10 segundos & Sistema identifica correctamente la intención & PASA \\
        \hline
        SYS-05 & Detección de pensamiento "verde" & Usuario piensa en color verde durante 10 segundos & Sistema identifica correctamente la intención & PASA \\
        \hline
        SYS-06 & Activación de dispositivo por pensamiento & Usuario piensa en color específico y se verifica actuación & Bombilla cambia al color detectado en <1s & PASA \\
        \hline
        SYS-07 & Operación continua & Sistema funciona por >30 minutos continuos & Sin degradación de rendimiento ni fugas de memoria & PASA \\
        \hline
        SYS-08 & Recuperación ante errores & Simular desconexión del dispositivo EEG durante operación & Sistema detecta error y permite reconexión & PASA \\
        \hline
    \end{tabular}
    \caption{Casos de prueba del sistema}
    \label{tab:system_tests}
\end{table}



\section{Pruebas de Seguridad}

Aunque para software de Clase A la norma no exige pruebas de seguridad exhaustivas, se han realizado verificaciones básicas para garantizar que el sistema no compromete la privacidad del usuario ni genera riesgos inaceptables.

\subsection{Gestión de Datos del Usuario}

Se ha verificado que:

\begin{itemize}
    \item Los datos EEG capturados se procesan localmente sin transmisión externa
    \item Los archivos de datos guardados se almacenan en formato anonimizado
    \item El sistema no recopila información personal identificable
\end{itemize}

\subsection{Seguridad Eléctrica}

El dispositivo BrainBit EEG utilizado en el sistema Neural Analytics cumple con los estándares FC y CE, garantizando su seguridad eléctrica y compatibilidad electromagnética. Dado que no se ha implementado hardware personalizado como parte de este proyecto, no se han requerido pruebas adicionales de seguridad eléctrica. No obstante, se ha verificado:

\begin{itemize}
    \item Ausencia de interacciones eléctricas peligrosas entre el dispositivo EEG y el sistema
    \item Comportamiento seguro durante la carga del dispositivo EEG
    \item Ausencia de sobrecalentamiento en operación prolongada
\end{itemize}

\section{Gestión de Anomalías}

Durante el proceso de pruebas se detectaron e implementaron correcciones para las siguientes anomalías significativas:

\begin{table}[ht]
    \centering
    \begin{tabular}{|c|p{6cm}|c|p{3.5cm}|}
        \hline
        \textbf{ID} & \textbf{Descripción} & \textbf{Severidad} & \textbf{Resolución} \\
        \hline
        AN-001 & Pérdida ocasional de datos EEG en sesiones prolongadas & Media & Implementación de buffer circular con retry automático \\
        \hline
        AN-002 & Falsos positivos en detección de señales con baja impedancia & Baja & Ajuste de umbral de confianza para clasificación \\
        \hline
        AN-003 & Latencia excesiva en interfaz gráfica durante visualización de señales & Media & Optimización del renderizado con muestreo adaptativo \\
        \hline
        AN-004 & Inconsistencia en la persistencia de configuraciones de usuario & Baja & Refactorización de sistema de almacenamiento de configuración \\
        \hline
        AN-005 & Problemas de reconexión con bombillas inteligentes tras pérdida de red & Media & Implementación de protocolo de reconexión resiliente \\
        \hline
    \end{tabular}
    \caption{Anomalías detectadas y resueltas}
    \label{tab:anomalies}
\end{table}

De conformidad con la sección 5.8.2 de la norma, todas las anomalías residuales conocidas se han documentado y evaluado, determinando que ninguna representa un riesgo inaceptable para el usuario.

\newpage
\section{Matriz de Trazabilidad}

Para garantizar la completitud de las pruebas, se ha desarrollado una matriz de trazabilidad que vincula los requisitos del sistema con los casos de prueba que los verifican:

\begin{table}[ht]
    \centering
    \begin{tabular}{|p{2cm}|p{5cm}|p{3cm}|p{3cm}|}
        \hline
        \textbf{Requisito} & \textbf{Descripción} & \textbf{Pruebas asociadas} & \textbf{Estado} \\
        \hline
        REQ-F01 & Adquisición de señales EEG desde BrainBit & UC-001, TC-001, SYS-02 & VERIFICADO \\
        \hline
        REQ-F02 & Clasificación de patrones cerebrales & UC-005, TC-003, SYS-04, SYS-05 & VERIFICADO \\
        \hline
        REQ-F03 & Control de dispositivos por pensamiento & UC-007, TC-006, SYS-06 & VERIFICADO \\
        \hline
        REQ-F04 & Visualización de estado del sistema & UC-010, TC-008, SYS-01 & VERIFICADO \\
        \hline
        REQ-F05 & Calibración del dispositivo EEG & UC-012, TC-010, SYS-03 & VERIFICADO \\
        \hline
        REQ-NF01 & Tiempo de respuesta < 3.5s & TC-020, SYS-06 & VERIFICADO \\
        \hline
        REQ-NF02 & Precisión de clasificación > 80\% & TC-021, SYS-04, SYS-05 & VERIFICADO \\
        \hline
        REQ-NF03 & Operación continua durante > 30min & TC-023, SYS-07 & VERIFICADO \\
        \hline
    \end{tabular}
    \caption{Matriz de trazabilidad de requisitos y pruebas}
    \label{tab:traceability}
\end{table}

\section{Conclusión de la Validación}

La validación del prototipo Neural Analytics ha seguido rigurosamente las directrices de la norma UNE-EN 62304:2007/A1:2016 para software de dispositivos médicos de Clase A. Se han completado satisfactoriamente los tres niveles de prueba requeridos (unitarias, integración y sistema), verificando el cumplimiento de todos los requisitos especificados.

Los resultados demuestran:

\begin{itemize}
    \item Alta precisión en la detección de intenciones del usuario (86.8\% promedio)
    \item Tiempos de respuesta dentro de los límites aceptables (3.1s promedio)
    \item Operación estable y robusta del sistema
\end{itemize}

Las anomalías detectadas durante el proceso de prueba han sido debidamente documentadas y corregidas, verificando que ninguna de las anomalías residuales conocidas representa un riesgo inaceptable para el usuario.

El proceso de pruebas ha aportado evidencia objetiva de que el sistema Neural Analytics cumple con los requisitos especificados y funciona correctamente en el entorno previsto de uso, quedando validado para su implementación como sistema de interfaz cerebro-computadora para el control de dispositivos domóticos mediante ondas cerebrales.