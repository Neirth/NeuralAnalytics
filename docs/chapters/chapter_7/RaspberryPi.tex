\chapter{Raspberry Pi 4 Model B (8GB)}

\section{Introducción}
    La Raspberry Pi 4 Model B \cite{raspberrypi4} es un ordenador de placa única (SBC) que destaca por su equilibrio entre rendimiento, consumo energético y facilidad de programación. Este modelo incorpora un procesador ARM de 64 bits, una cantidad significativa de memoria RAM y diversas interfaces de conectividad, características que lo convierten en una opción atractiva para sistemas embebidos y aplicaciones de control. Su arquitectura ARM cuenta con amplio soporte por parte de los principales fabricantes de sistemas operativos, incluyendo distribuciones Linux empresariales y sistemas operativos en tiempo real.

\section{Especificaciones Técnicas}
    El modelo de 8GB de la Raspberry Pi 4 se basa en la siguiente arquitectura de hardware:

    \subsection{Procesador y Memoria}
    \begin{itemize}
        \item CPU: Quad-Core ARM Cortex-A72 (64 bits) a 1.5GHz.
        \item GPU: VideoCore VI compatible con OpenGL ES 3.0.
        \item Memoria RAM: 8 GB LPDDR4 SDRAM.
    \end{itemize}

    \subsection{Requerimientos de Energía}
    La Raspberry Pi 4 Model B requiere una fuente de alimentación de 5V y 3A a través de un puerto USB-C. Para configuraciones que utilicen dispositivos USB adicionales, se recomienda una fuente con mayor capacidad de corriente.

    \newpage
    \subsection{Interfaces y Conectividad}
    \begin{itemize}
        \item Red:
        \begin{itemize}
            \item Gigabit Ethernet (compatible con PoE mediante un módulo adicional).
            \item Wi-Fi 802.11 b/g/n/ac de doble banda (2.4 GHz y 5.0 GHz).
            \item Bluetooth 5.0 con BLE.
        \end{itemize}
        \item Almacenamiento:
        \begin{itemize}
            \item Ranura para tarjeta microSD.
        \end{itemize}
        \item Puertos USB:
        \begin{itemize}
            \item 2 puertos USB 3.0.
            \item 2 puertos USB 2.0.
        \end{itemize}
        \item Vídeo y Audio:
        \begin{itemize}
            \item 2 puertos micro-HDMI con soporte hasta 4K a 60Hz.
            \item Salida de audio analógico y vídeo compuesto mediante conector TRRS de 3.5 mm.
        \end{itemize}
        \item Expansión:
        \begin{itemize}
            \item Conector GPIO de 40 pines compatible con modelos anteriores.
            \item Conector CSI para cámaras.
            \item Conector DSI para pantallas.
        \end{itemize}
    \end{itemize}

    \subsection{Consideraciones Térmicas}
    El sistema de gestión térmica de la Raspberry Pi 4 permite reducir la frecuencia y el voltaje del procesador en situaciones de baja carga para minimizar el consumo de energía y la generación de calor. En cargas elevadas y entornos de temperatura alta, se recomienda el uso de sistemas de disipación adicionales, como disipadores o ventiladores, para mantener la estabilidad operativa.

\section{Aplicaciones}
    La Raspberry Pi 4 Model B (8GB) resulta especialmente adecuada para:
    \begin{itemize}
        \item Sistemas embebidos de control y automatización.
        \item Aplicaciones IoT y monitorización de sensores.
        \item Dispositivos edge computing de bajo consumo.
        \item Prototipos y desarrollo de sistemas embebidos.
        \item Servidores domésticos y sistemas de control del hogar.
    \end{itemize}

\section{Elección de este dispositivo para el Proyecto}
    El modelo de 8GB de la Raspberry Pi 4 representa una solución versátil y compacta para el desarrollo del sistema de control domótico propuesto. Su amplia capacidad de memoria RAM y su rendimiento equilibrado permiten ejecutar aplicaciones complejas y procesos en tiempo real con eficiencia. Además, la compatibilidad con sistemas operativos en tiempo real y distribuciones Linux empresariales garantiza una base sólida para el desarrollo y la implementación del sistema.