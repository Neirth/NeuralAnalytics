\chapter{Raspberry Pi 4 Model B (8GB)}

\section{Introducción}
    Elegir la Raspberry Pi 4 Model B \cite{raspberrypi4} para este proyecto no fue inmediato, la verdad. Inicialmente había considerado otras opciones más potentes —incluso estuve mirando algunos mini-PCs x86—, pero al final este ordenador de placa única me acabó convenciendo por razones bastante prácticas. El equilibrio entre rendimiento, consumo y facilidad de desarrollo resultó ser exactamente lo que necesitaba, aunque admito que esto lo confirmé solo después de varias semanas trabajando con él y de algunos dolores de cabeza iniciales.

    Lo que realmente me decidió fue descubrir que incorpora un procesador ARM de 64 bits que funciona sorprendentemente bien (mejor de lo que esperaba, sinceramente), memoria RAM más que suficiente —especialmente en el modelo de 8GB que terminé eligiendo después de dudar mucho entre este y el de 4GB—, y prácticamente todas las interfaces que requería mi diseño. Lo que más tranquilidad me dio, sin embargo, fue verificar que su arquitectura ARM tiene un soporte bastante sólido por parte de los principales fabricantes. Esto era crucial porque ya había tenido experiencias bastante frustrantes con incompatibilidades en proyectos anteriores, y francamente no quería repetir esa experiencia.

\section{Especificaciones Técnicas}
    El modelo de 8GB de la Raspberry Pi 4 se basa en una arquitectura de hardware que me sorprendió gratamente por su relación prestaciones-precio, aunque reconozco que mis expectativas iniciales no eran demasiado altas:

    \subsection{Procesador y Memoria}
    \begin{itemize}
        \item CPU: Quad-Core ARM Cortex-A72 (64 bits) a 1.5GHz —aunque en realidad puede hacer boost hasta 1.8GHz en ciertas condiciones, algo que descubrí más tarde.
        \item GPU: VideoCore VI compatible con OpenGL ES 3.0 —suficiente para mis necesidades, aunque no esperes milagros gráficos.
        \item Memoria RAM: 8 GB LPDDR4 SDRAM. Este fue el factor que realmente me convenció de invertir un poco más en este modelo específico, especialmente después de haberme quedado corto de memoria en proyectos anteriores.
    \end{itemize}

    \subsection{Requerimientos de Energía}
    La Raspberry Pi 4 Model B requiere una fuente de alimentación de 5V y 3A a través de USB-C. Para configuraciones que incluyan dispositivos USB adicionales, recomiendo usar una fuente con mayor capacidad —algo que aprendí por experiencia tras algunos problemas de estabilidad bastante inesperados durante las primeras pruebas. Tener esto en cuenta es fundamental, especialmente en un proyecto como este donde la estabilidad es clave. De hecho, me pasé una tarde entera pensando que tenía un problema de software cuando en realidad era simplemente que la fuente que estaba usando no daba suficiente corriente para todo el sistema.

    \subsection{Interfaces y Conectividad}
    \begin{itemize}
        \item Red:
        \begin{itemize}
            \item Gigabit Ethernet (compatible con PoE mediante un módulo adicional, aunque no lo implementé —quizás debería haberlo considerado más seriamente).
            \item Wi-Fi 802.11 b/g/n/ac de doble banda (2.4 GHz y 5.0 GHz). En mi experiencia funciona bastante bien, aunque la señal no es tan potente como me hubiera gustado inicialmente.
            \item Bluetooth 5.0 con BLE, que me resultó perfecto para conectar con el dispositivo BrainBit —una de esas cosas que funcionó mejor de lo esperado desde el primer momento.
        \end{itemize}
        \item Almacenamiento:
        \begin{itemize}
            \item Ranura para tarjeta microSD. Recomiendo encarecidamente invertir en una tarjeta de calidad —las opciones baratas me dieron más de un problema de rendimiento y algún que otro susto con corrupción de datos.
        \end{itemize}
        \item Puertos USB:
        \begin{itemize}
            \item 2 puertos USB 3.0.
            \item 2 puertos USB 2.0.
        \end{itemize}
        \item Vídeo y Audio:
        \begin{itemize}
            \item 2 puertos micro-HDMI con soporte hasta 4K a 60Hz.
            \item Salida de audio analógico y vídeo compuesto mediante conector TRRS de 3.5 mm.
        \end{itemize}
        \item Expansión:
        \begin{itemize}
            \item Conector GPIO de 40 pines compatible con modelos anteriores.
            \item Conector CSI para cámaras.
            \item Conector DSI para pantallas.
        \end{itemize}
    \end{itemize}

    \subsection{Consideraciones Térmicas}
    El sistema de gestión térmica de la Raspberry Pi 4 reduce automáticamente la frecuencia y el voltaje del procesador durante periodos de baja demanda, lo cual funciona bien para ahorrar energía y evitar sobrecalentamiento. Ahora bien, bajo cargas intensivas y especialmente cuando hace calor, necesitas añadir disipación adicional como disipadores o ventiladores. Esto lo confirmé de la forma más incómoda posible: cuando se me empezó a resetear durante las primeras sesiones de pruebas intensivas y me costó un rato darme cuenta de que era un problema térmico y no de software.

\section{Elección de este dispositivo para el Proyecto}
    El modelo de 8GB de la Raspberry Pi 4 resultó ser una solución bastante versátil y compacta para lo que necesitaba desarrollar, aunque no voy a mentir: hubo momentos en los que me planteé si no habría sido mejor optar por algo más potente. Tener 8GB de RAM me dio el margen suficiente para ejecutar aplicaciones complejas sin preocuparme constantemente por limitaciones de memoria, y el rendimiento general demostró ser adecuado para procesos en tiempo real —al menos para lo que mi proyecto específico requería.
    
    Lo que más valoré al final fue la compatibilidad con sistemas operativos en tiempo real y las distribuciones Linux que ya conocía. Poder usar herramientas familiares me aceleró considerablemente el desarrollo, algo que no había anticipado completamente al principio pero que resultó ser una ventaja importante cuando los plazos empezaron a estrechar.