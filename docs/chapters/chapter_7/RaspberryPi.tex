\chapter{Raspberry Pi 4 Model B (8GB)}

\section{Introducción}

La selección de la Raspberry Pi 4 Model B \cite{raspberrypi4} se fundamentó en un análisis comparativo de múltiples opciones de hardware, incluyendo mini-PCs x86 y otras plataformas embebidas. Este ordenador de placa única constituye una solución apropiada por razones técnicas y económicas. El equilibrio entre rendimiento, consumo energético y facilidad de desarrollo satisface los requerimientos específicos de aplicaciones embebidas de tiempo real.

Los factores determinantes incluyen un procesador ARM de 64 bits con eficiencia energética optimizada, una cantidad de memoria RAM de 8GB que garantiza el margen operativo necesario, y la disponibilidad de interfaces requeridas por el diseño del sistema. La arquitectura ARM cuenta con soporte robusto por parte de los principales fabricantes de software, aspecto fundamental para garantizar la compatibilidad con las librerías y herramientas utilizadas en el desarrollo de aplicaciones embebidas.

\section{Especificaciones Técnicas}

El modelo de 8GB de la Raspberry Pi 4 se basa en una arquitectura de hardware que ofrece una relación prestaciones-precio adecuada para aplicaciones embebidas de tiempo real:

\subsection{Procesador y Memoria}

\begin{itemize}
    \item CPU: Quad-Core ARM Cortex-A72 (64 bits) a 1.5GHz, con capacidad de alcanzar hasta 1.8GHz mediante overclocking.
    \item GPU: VideoCore VI compatible con OpenGL ES 3.0, apropiada para aplicaciones que requieren aceleración gráfica.
    \item Memoria RAM: 8 GB LPDDR4 SDRAM. Esta configuración de memoria proporciona el margen operativo necesario para aplicaciones en tiempo real complejas.
\end{itemize}

\subsection{Requerimientos de Energía}

La Raspberry Pi 4 Model B requiere una fuente de alimentación de 5V y 3A a través de USB-C. Para configuraciones que incluyan dispositivos USB adicionales, se recomienda el uso de una fuente con mayor capacidad para garantizar la estabilidad del sistema. La correcta especificación de la fuente de alimentación constituye un aspecto fundamental en sistemas donde la estabilidad operacional es crítica. La insuficiencia de corriente puede manifestarse como inestabilidades aparentemente relacionadas con software cuando en realidad tienen origen en limitaciones de la alimentación.

\subsection{Interfaces y Conectividad}

\begin{itemize}
    \item Red:
    \begin{itemize}
        \item Gigabit Ethernet (compatible con PoE mediante un módulo adicional).
        \item Wi-Fi 802.11 b/g/n/ac de doble banda (2.4 GHz y 5.0 GHz) con rendimiento adecuado para aplicaciones embebidas.
        \item Bluetooth 5.0 con BLE, apropiado para la conexión con dispositivos de monitorización biométrica.
    \end{itemize}
    \item Almacenamiento:
    \begin{itemize}
        \item Ranura para tarjeta microSD. Se recomienda el uso de tarjetas de alta calidad para garantizar rendimiento y fiabilidad en aplicaciones críticas.
    \end{itemize}
    \item Puertos USB:
    \begin{itemize}
        \item 2 puertos USB 3.0.
        \item 2 puertos USB 2.0.
    \end{itemize}
    \item Vídeo y Audio:
    \begin{itemize}
        \item 2 puertos micro-HDMI con soporte hasta 4K a 60Hz.
        \item Salida de audio analógico y vídeo compuesto mediante conector TRRS de 3.5 mm.
    \end{itemize}
    \item Expansión:
    \begin{itemize}
        \item Conector GPIO de 40 pines compatible con modelos anteriores.
        \item Conector CSI para cámaras.
        \item Conector DSI para pantallas.
    \end{itemize}
\end{itemize}

\subsection{Consideraciones Térmicas}

El sistema de gestión térmica de la Raspberry Pi 4 reduce automáticamente la frecuencia y el voltaje del procesador durante periodos de baja demanda, operando eficientemente para el ahorro energético y la prevención del sobrecalentamiento. Bajo cargas intensivas y en condiciones de temperatura ambiente elevada, se requiere disipación adicional mediante disipadores o ventiladores. La gestión térmica inadecuada puede manifestarse como inestabilidades del sistema que requieren identificación precisa del origen del problema para su resolución efectiva.

\section{Elección de este dispositivo}

El modelo de 8GB de la Raspberry Pi 4 constituye una solución versátil y compacta que satisface los requerimientos técnicos de aplicaciones embebidas de tiempo real. La configuración de 8GB de RAM proporciona el margen operativo suficiente para ejecutar aplicaciones complejas sin limitaciones de memoria, mientras que el rendimiento general cumple con las exigencias de procesamiento en tiempo real.

La compatibilidad con sistemas operativos en tiempo real y distribuciones Linux estándar facilita la integración con el ecosistema de herramientas de desarrollo requerido. Esta compatibilidad optimiza el proceso de desarrollo al permitir el uso de herramientas y bibliotecas consolidadas, aspectos fundamentales para el cumplimiento de objetivos técnicos en aplicaciones embebidas de alta disponibilidad.