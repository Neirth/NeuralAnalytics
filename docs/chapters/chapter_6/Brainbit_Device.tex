\chapter{BrainBit Headset}

\section{Introducci\'on}
La selección del dispositivo EEG constituye un factor determinante en el desarrollo de interfaces cerebro-computadora. Tras la evaluación de múltiples opciones disponibles en el mercado, se seleccionó el \textbf{BrainBit} \cite{brainbit} como solución óptima, fundamentada en criterios que exceden las especificaciones técnicas básicas.

El proceso de selección priorizó la identificación de un dispositivo que combinara precisión en la adquisición de señales con facilidad de uso en entornos reales. Los criterios establecidos descartaron equipos limitados a condiciones de laboratorio controladas, requiriendo un sistema capaz de operar eficientemente en condiciones variables sin procedimientos de configuración complejos. La decisión final se fundamentó en el equilibrio entre calidad científica y practicidad operacional, características esenciales para el desarrollo de aplicaciones de neurocontrol efectivas.

\section{Características Técnicas del Dispositivo}
El BrainBit es un casco EEG portátil que utiliza \textbf{electrodos secos}. Esta característica elimina la necesidad de gel conductor, optimizando los procedimientos de adquisición de datos y facilitando mediciones repetidas sin preparación adicional de electrodos.

Las especificaciones de mayor relevancia para el proyecto fueron:

    \begin{itemize}
        \item \textbf{Canales EEG}: 4 canales (T3, T4, O1, O2).
        \item \textbf{Frecuencia de muestreo}: 250 Hz.
        \item \textbf{Interfaz de comunicación}: Bluetooth Low Energy (BLE).
        \item \textbf{Tiempo de uso continuo}: Hasta 12 horas.
        \item \textbf{Filtro de ruido integrado}: Reduce la necesidad de procesamiento adicional de señales.
        \item \textbf{Ubicación de electrodos}: Sistema 10-20 estándar, con sensores en \textbf{O1 y O2} para actividad occipital.
    \end{itemize}

La configuración específica de los electrodos O1 y O2, junto con los electrodos T3 y T4, constituye un factor determinante en la capacidad del dispositivo para detectar patrones visuales relacionados con colores. El posicionamiento sobre las regiones occipital y temporal proporciona acceso directo a las áreas cerebrales responsables del procesamiento visual, cumpliendo los requerimientos técnicos para aplicaciones de neurocontrol basadas en estimulación visual.

\section{Metodologías para la Adquisición y Procesamiento}
El desarrollo técnico se estructuró en tres fases diferenciadas, cada una abordando aspectos específicos del procesamiento de señales EEG para aplicaciones de neurocontrol.

    \subsection{Captación de Señales}
    La colocación precisa de los electrodos \textbf{O1 y O2} sobre la región occipital constituye un aspecto crítico para la calidad de la adquisición de señales. El establecimiento de un protocolo de posicionamiento requiere múltiples sesiones experimentales para garantizar la reproducibilidad y consistencia de las mediciones. La literatura científica describe estos procedimientos como estándar, aunque su implementación práctica demanda destreza técnica específica.
    
    El SDK proporcionado por el fabricante ofrece funcionalidades básicas de captura, sin embargo, las exigencias específicas del proyecto requieren el desarrollo de una interfaz personalizada. Las herramientas estándar se orientan hacia aplicaciones genéricas de monitorización del estado cognitivo, mientras que este proyecto demanda capacidades especializadas en el procesamiento de patrones visuales. En consecuencia, se implementó una interfaz propia que se integra eficientemente con el sistema de procesamiento y clasificación desarrollado.
    
    Para garantizar la estabilidad en la conexión de los electrodos, se desarrolló una interfaz de calibración que monitoriza durante dicho proceso la calidad del contacto de cada electrodo. Esta herramienta optimiza la calidad de las sesiones experimentales, cuyos detalles de implementación se presentan en el Capítulo~\ref{ch:core_implementing}.

    \subsection{Acondicionamiento de Señales}
    El BrainBit incorpora un sistema de filtrado integrado de calidad, y la propia calidad base de las señales proporcionadas por el dispositivo permite mantener estos procesamientos en un nivel de complejidad moderado, evitando la implementación de algoritmos de filtrado complejos característicos del procesamiento avanzado de señales EEG.
    
    El acondicionamiento se centra en la normalización de amplitudes y la segmentación temporal apropiada para el análisis posterior. Estos procesos, fundamentales para asegurar la consistencia de los datos de entrada al sistema, requieren ajustes precisos para optimizar la calidad de la clasificación posterior.

    \subsection{Análisis mediante Aprendizaje Profundo}
    En esta fase, se implementaron modelos de \textbf{aprendizaje profundo} especializados en el procesamiento de series temporales EEG. Estos modelos fueron entrenados para identificar patrones específicos relacionados con la visualización mental de colores —concretamente rojo y verde—.
    
    Los fundamentos teóricos de estos modelos se explican en detalle en el Capítulo~\ref{ch:deep_learning_models}, y todo lo concerniente al entrenamiento práctico se describe en el Capítulo~\ref{ch:model_training}.

\section{Campos de Aplicación}
La tecnología de interfaces cerebro-computadora basada en señales EEG presenta múltiples campos de aplicación potencial, aunque muchos se encuentran en fase de investigación y desarrollo:

    \begin{itemize}
        \item Desarrollo de interfaces cerebro-máquina para asistir a personas con diversidad funcional. Esta aplicación habilita nuevas modalidades de comunicación e interacción tecnológica, representando un área de investigación con potencial significativo para mejorar la calidad de vida.
        \item Sistemas de control en entornos estériles médicos e industriales donde el contacto físico compromete la asepsia o la seguridad operacional. Este campo presenta viabilidad técnica inmediata dado el estado actual de la tecnología.
        \item Aplicaciones en realidad aumentada y virtual que permiten modalidades de interacción más naturales e intuitivas. El desarrollo de estas aplicaciones plantea consideraciones sobre la ergonomía cognitiva y la conveniencia de diferentes modalidades de control.
    \end{itemize}

La consideración de estos campos de aplicación proporciona perspectiva sobre el potencial impacto de las interfaces cerebro-computadora en diversos contextos tecnológicos y sociales. El desarrollo de prototipos funcionales constituye un paso fundamental hacia la materialización de aplicaciones prácticas en estos dominios específicos.
